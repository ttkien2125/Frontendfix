\documentclass[a4paper,12pt]{article}
\usepackage[utf8]{vietnam}
\usepackage[T5]{fontenc}
\usepackage{geometry}
\usepackage{array}
\usepackage{longtable}
\usepackage{xcolor}
\usepackage{colortbl}
\usepackage{amssymb}

\geometry{left=2cm,right=2cm,top=2cm,bottom=2cm}

\definecolor{headerblue}{RGB}{0,102,204}

\newcommand{\testcase}[9]{
\begin{longtable}{|p{4cm}|p{11cm}|}
\hline
\rowcolor{gray!10}
\textcolor{headerblue}{\textbf{Test No.}} & \textbf{#1} \\
\hline
\textcolor{headerblue}{\textbf{Current status}} & $\square$ Passed \quad $\square$ Failed \quad $\square$ Pending \\
\hline
\textcolor{headerblue}{\textbf{Title}} & \textit{#2} \\
\hline
\textcolor{headerblue}{\textbf{Description}} & #3 \\
\hline
\textcolor{headerblue}{\textbf{Approach}} & #4 \\
\hline
\multicolumn{2}{|c|}{\cellcolor{gray!10}\textcolor{headerblue}{\textbf{Test Steps}}} \\
\hline
\multicolumn{2}{|c|}{
\begin{tabular}{|p{1cm}|p{4cm}|p{3cm}|p{3.5cm}|p{2cm}|}
\hline
\rowcolor{gray!20}
\textbf{Step} & \textbf{Action} & \textbf{Purpose} & \textbf{Expected result} & \textbf{Comment} \\
\hline
#5
\end{tabular}
} \\
\hline
\textcolor{headerblue}{\textbf{Concluding remark}} & #6 \\
\hline
\textcolor{headerblue}{\textbf{Testing team}} & #7 \\
\hline
\textcolor{headerblue}{\textbf{Date completed}} & #8 \\
\hline
\end{longtable}
\vspace{1cm}
}

\begin{document}

\title{\textbf{BÁO CÁO TEST CASES} \\ \Large Hệ thống Quản lý Tòa nhà BlueMoon}
\author{Nhóm Phát triển BlueMoon}
\date{\today}
\maketitle

\newpage
\tableofcontents
\newpage

\section{Giới thiệu}

Tài liệu này mô tả các test cases cho hệ thống quản lý tòa nhà BlueMoon. Hệ thống được phát triển với backend Python/FastAPI và frontend React/TypeScript, hỗ trợ 4 loại người dùng: Cư dân (Resident), Kế toán (Accountant), Quản lý (Manager), và Quản trị viên (Admin).

\subsection{Mục đích kiểm thử}
\begin{itemize}
    \item Đảm bảo tính năng hoạt động đúng theo yêu cầu
    \item Kiểm tra phân quyền truy cập theo vai trò
    \item Xác minh tính toàn vẹn của dữ liệu
    \item Đánh giá trải nghiệm người dùng
\end{itemize}

\subsection{Phạm vi kiểm thử}
\begin{itemize}
    \item Xác thực và phân quyền
    \item Quản lý hóa đơn và thanh toán
    \item Quản lý thông báo
    \item Quản lý chỉ số công tơ
    \item Quản lý căn hộ và cư dân
    \item Tính năng kế toán
\end{itemize}

\newpage
\section{Test Cases - Xác thực và Phân quyền}

\testcase
{TC-001}
{Đăng nhập với thông tin hợp lệ}
{Kiểm tra chức năng đăng nhập của người dùng với tài khoản và mật khẩu hợp lệ}
{Functional testing - Kiểm thử chức năng cơ bản của hệ thống xác thực}
{
1 & Truy cập trang đăng nhập & Mở giao diện đăng nhập & Hiển thị form đăng nhập với các trường username và password & \\
\hline
2 & Nhập username hợp lệ & Kiểm tra validation & Chấp nhận input & \\
\hline
3 & Nhập password hợp lệ & Kiểm tra validation & Chấp nhận input & \\
\hline
4 & Nhấn nút "Đăng nhập" & Xác thực thông tin & Chuyển đến dashboard tương ứng với role & \\
\hline
5 & Kiểm tra session & Xác nhận trạng thái đăng nhập & Token được lưu trong localStorage & \\
\hline
}
{Test case pass khi người dùng đăng nhập thành công và được chuyển đến dashboard phù hợp với vai trò của họ}
{Lead: Nguyễn Văn A, Members: Trần Thị B, Lê Văn C}
{03/01/2026}

\testcase
{TC-002}
{Đăng nhập với thông tin không hợp lệ}
{Kiểm tra xử lý lỗi khi người dùng nhập sai thông tin đăng nhập}
{Negative testing - Kiểm thử trường hợp thất bại}
{
1 & Truy cập trang đăng nhập & Mở giao diện đăng nhập & Hiển thị form đăng nhập & \\
\hline
2 & Nhập username sai & Kiểm tra validation & Chấp nhận input & \\
\hline
3 & Nhập password sai & Kiểm tra validation & Chấp nhận input & \\
\hline
4 & Nhấn nút "Đăng nhập" & Xác thực thông tin & Hiển thị thông báo lỗi "Tên đăng nhập hoặc mật khẩu không đúng" & \\
\hline
5 & Kiểm tra session & Xác nhận không tạo session & Không có token trong localStorage & \\
\hline
}
{Test case pass khi hệ thống hiển thị thông báo lỗi phù hợp và không cho phép truy cập}
{Lead: Nguyễn Văn A, Members: Trần Thị B}
{03/01/2026}

\testcase
{TC-003}
{Kiểm tra phân quyền truy cập theo vai trò}
{Xác minh rằng mỗi vai trò chỉ có thể truy cập các chức năng được phép}
{Security testing - Kiểm thử bảo mật và phân quyền}
{
1 & Đăng nhập với tài khoản Resident & Kiểm tra quyền truy cập & Dashboard hiển thị: Tổng quan, Hóa đơn, Thanh toán, Thông báo, Tài khoản & \\
\hline
2 & Thử truy cập URL của Accountant & Kiểm tra bảo mật & Hệ thống từ chối truy cập & \\
\hline
3 & Đăng nhập với tài khoản Accountant & Kiểm tra quyền truy cập & Dashboard hiển thị: Tổng quan, Thanh toán ngoại tuyến, Quản lý kế toán, Biên lai, Tài khoản & \\
\hline
4 & Thử truy cập chức năng Admin & Kiểm tra bảo mật & Hệ thống từ chối truy cập & \\
\hline
5 & Kiểm tra với Manager và Admin & Đảm bảo đầy đủ & Mỗi role có quyền phù hợp & \\
\hline
}
{Test case pass khi mỗi vai trò chỉ truy cập được các chức năng được phân quyền}
{Lead: Lê Văn C, Members: Phạm Thị D, Hoàng Văn E}
{03/01/2026}

\newpage
\section{Test Cases - Chức năng Cư dân (Resident)}

\testcase
{TC-004}
{Cư dân xem danh sách hóa đơn}
{Kiểm tra chức năng hiển thị danh sách hóa đơn của cư dân}
{Functional testing - Kiểm thử hiển thị dữ liệu}
{
1 & Đăng nhập với tài khoản Resident & Truy cập hệ thống & Dashboard cư dân được hiển thị & \\
\hline
2 & Nhấn vào tab "Hóa đơn" & Mở trang hóa đơn & Hiển thị danh sách hóa đơn theo tháng & \\
\hline
3 & Kiểm tra thông tin hóa đơn & Xác minh dữ liệu & Hiển thị đầy đủ: Tháng, tổng tiền, trạng thái thanh toán & \\
\hline
4 & Lọc hóa đơn theo trạng thái & Kiểm tra filter & Danh sách cập nhật theo bộ lọc (Tất cả/Chưa thanh toán/Đã thanh toán) & \\
\hline
5 & Xem chi tiết hóa đơn & Kiểm tra dialog & Modal hiển thị chi tiết các khoản phí & \\
\hline
}
{Test case pass khi cư dân có thể xem đầy đủ thông tin hóa đơn với các bộ lọc hoạt động chính xác}
{Lead: Nguyễn Văn A, Members: Trần Thị B}
{03/01/2026}

\testcase
{TC-005}
{Cư dân xem lịch sử thanh toán}
{Kiểm tra chức năng hiển thị lịch sử các giao dịch thanh toán}
{Functional testing - Kiểm thử hiển thị lịch sử}
{
1 & Đăng nhập với tài khoản Resident & Truy cập hệ thống & Dashboard cư dân được hiển thị & \\
\hline
2 & Nhấn vào tab "Thanh toán" & Mở trang thanh toán & Hiển thị lịch sử thanh toán & \\
\hline
3 & Kiểm tra thông tin giao dịch & Xác minh dữ liệu & Hiển thị: Ngày thanh toán, số tiền, phương thức, trạng thái & \\
\hline
4 & Tìm kiếm theo ngày & Kiểm tra filter & Danh sách được lọc theo khoảng thời gian & \\
\hline
5 & Xem chi tiết giao dịch & Kiểm tra dialog & Modal hiển thị thông tin đầy đủ của giao dịch & \\
\hline
}
{Test case pass khi cư dân có thể xem đầy đủ lịch sử thanh toán với các công cụ tìm kiếm hoạt động tốt}
{Lead: Trần Thị B, Members: Lê Văn C}
{03/01/2026}

\testcase
{TC-006}
{Cư dân xem và quản lý thông báo}
{Kiểm tra chức năng hiển thị và quản lý thông báo của cư dân}
{Functional testing - Kiểm thử chức năng thông báo}
{
1 & Đăng nhập với tài khoản Resident & Truy cập hệ thống & Dashboard cư dân được hiển thị & \\
\hline
2 & Nhấn vào tab "Thông báo" & Mở trang thông báo & Hiển thị danh sách thông báo với stats (Tổng/Chưa đọc/Đã đọc) & \\
\hline
3 & Kiểm tra phân loại thông báo & Xác minh icon & Thông báo điện có icon lightning, nước có icon droplet & \\
\hline
4 & Lọc thông báo chưa đọc & Kiểm tra filter & Hiển thị chỉ thông báo chưa đọc với nền gradient xanh & \\
\hline
5 & Đánh dấu 1 thông báo đã đọc & Kiểm tra cập nhật & Thông báo chuyển sang trạng thái đã đọc, toast hiển thị & \\
\hline
6 & Đánh dấu tất cả đã đọc & Kiểm tra bulk action & Tất cả thông báo chuyển sang đã đọc, số lượng unread = 0 & \\
\hline
7 & Làm mới danh sách & Kiểm tra refresh & Danh sách được tải lại từ server & \\
\hline
}
{Test case pass khi tất cả chức năng thông báo hoạt động chính xác với UI phản hồi nhanh}
{Lead: Nguyễn Văn A, Members: Phạm Thị D}
{03/01/2026}

\newpage
\section{Test Cases - Chức năng Kế toán (Accountant)}

\testcase
{TC-007}
{Kế toán nhập chỉ số công tơ thủ công}
{Kiểm tra chức năng nhập chỉ số công tơ điện và nước cho căn hộ}
{Functional testing - Kiểm thử nhập liệu}
{
1 & Đăng nhập với tài khoản Accountant & Truy cập hệ thống & Dashboard kế toán được hiển thị & \\
\hline
2 & Nhấn vào tab "Quản lý kế toán" & Mở trang kế toán & Hiển thị các tabs: Chỉ số công tơ, Phí dịch vụ, Hóa đơn & \\
\hline
3 & Chọn tab "Chỉ số công tơ" & Mở form nhập & Hiển thị form với dropdown căn hộ và input điện/nước & \\
\hline
4 & Chọn căn hộ và tháng & Kiểm tra validation & Form chấp nhận input hợp lệ & \\
\hline
5 & Nhập chỉ số điện và nước & Kiểm tra validation & Chỉ chấp nhận số dương & \\
\hline
6 & Nhấn "Lưu chỉ số" & Thực hiện lưu & Hiển thị toast thành công, danh sách cập nhật & \\
\hline
7 & Kiểm tra dữ liệu đã lưu & Xác minh & Chỉ số xuất hiện trong bảng với thông tin chính xác & \\
\hline
}
{Test case pass khi chỉ số công tơ được lưu thành công và hiển thị chính xác}
{Lead: Lê Văn C, Members: Hoàng Văn E}
{03/01/2026}

\testcase
{TC-008}
{Kế toán import chỉ số công tơ từ CSV}
{Kiểm tra chức năng import hàng loạt chỉ số công tơ từ file CSV}
{Functional testing - Kiểm thử import dữ liệu}
{
1 & Mở tab "Chỉ số công tơ" & Chuẩn bị import & Hiển thị nút "Import từ CSV" & \\
\hline
2 & Nhấn "Import từ CSV" & Mở dialog upload & Dialog hiển thị hướng dẫn format CSV & \\
\hline
3 & Chọn file CSV hợp lệ & Upload file & File được chấp nhận & \\
\hline
4 & Nhấn "Xác nhận import" & Xử lý dữ liệu & Hệ thống xử lý từng dòng trong CSV & \\
\hline
5 & Kiểm tra kết quả import & Xác minh dữ liệu & Hiển thị số lượng thành công/thất bại, toast thông báo & \\
\hline
6 & Kiểm tra dữ liệu trong bảng & Xác minh hiển thị & Tất cả chỉ số từ CSV xuất hiện trong danh sách & \\
\hline
}
{Test case pass khi import CSV thành công với thông báo chính xác về số lượng}
{Lead: Lê Văn C, Members: Trần Thị B, Phạm Thị D}
{03/01/2026}

\testcase
{TC-009}
{Kế toán quản lý phí dịch vụ}
{Kiểm tra chức năng tạo và cập nhật phí dịch vụ}
{Functional testing - Kiểm thử CRUD}
{
1 & Mở tab "Phí dịch vụ" & Truy cập quản lý phí & Hiển thị danh sách phí dịch vụ hiện có & \\
\hline
2 & Nhấn "Thêm phí dịch vụ" & Mở dialog tạo mới & Form hiển thị các trường: Tên, đơn giá, đơn vị & \\
\hline
3 & Điền thông tin phí mới & Kiểm tra validation & Form validation các trường bắt buộc & \\
\hline
4 & Nhấn "Lưu" & Tạo phí mới & Phí được tạo, hiển thị toast thành công & \\
\hline
5 & Chỉnh sửa phí vừa tạo & Kiểm tra update & Dialog hiển thị thông tin cũ, cho phép chỉnh sửa & \\
\hline
6 & Cập nhật đơn giá & Kiểm tra save & Đơn giá được cập nhật, danh sách refresh & \\
\hline
}
{Test case pass khi có thể tạo, xem, và cập nhật phí dịch vụ thành công}
{Lead: Hoàng Văn E, Members: Nguyễn Văn A}
{03/01/2026}

\testcase
{TC-010}
{Kế toán tính toán hóa đơn tự động}
{Kiểm tra chức năng tính toán và tạo hóa đơn tự động cho căn hộ}
{Functional testing - Kiểm thử tính toán nghiệp vụ}
{
1 & Mở tab "Hóa đơn" trong Quản lý kế toán & Truy cập tính hóa đơn & Hiển thị form tính hóa đơn & \\
\hline
2 & Chọn tháng/năm để tính & Chọn kỳ tính & Dropdown hiển thị các tháng có thể chọn & \\
\hline
3 & Nhấn "Tính toán hóa đơn" & Thực hiện tính toán & Hệ thống tính toán dựa trên chỉ số công tơ và phí dịch vụ & \\
\hline
4 & Kiểm tra kết quả & Xác minh logic & Hiển thị số lượng hóa đơn được tạo thành công & \\
\hline
5 & Xem chi tiết hóa đơn & Kiểm tra accuracy & Các khoản phí được tính chính xác (điện, nước, phí dịch vụ) & \\
\hline
6 & Kiểm tra trong tab Biên lai & Xác minh tích hợp & Hóa đơn mới xuất hiện với trạng thái "Chưa thanh toán" & \\
\hline
}
{Test case pass khi hóa đơn được tính toán chính xác dựa trên công thức nghiệp vụ}
{Lead: Lê Văn C, Members: Hoàng Văn E, Phạm Thị D}
{03/01/2026}

\testcase
{TC-011}
{Kế toán xử lý thanh toán ngoại tuyến bằng tiền mặt}
{Kiểm tra chức năng ghi nhận thanh toán tiền mặt cho hóa đơn}
{Functional testing - Kiểm thử xử lý thanh toán}
{
1 & Nhấn vào tab "Thanh toán ngoại tuyến" & Truy cập trang & Hiển thị dropdown tìm căn hộ & \\
\hline
2 & Tìm và chọn căn hộ & Lấy dữ liệu hóa đơn & Hiển thị danh sách hóa đơn chưa thanh toán & \\
\hline
3 & Kiểm tra thông tin hóa đơn & Xác minh hiển thị & Hiển thị tháng, tổng tiền, chi tiết các khoản phí & \\
\hline
4 & Chọn hóa đơn cần thanh toán & Đánh dấu chọn & Checkbox được chọn, tổng tiền được tính & \\
\hline
5 & Nhấn "Thanh toán bằng tiền mặt" & Mở dialog xác nhận & Dialog hiển thị tổng tiền cần thu & \\
\hline
6 & Xác nhận thanh toán & Xử lý giao dịch & Toast thành công, hóa đơn chuyển sang "Đã thanh toán" & \\
\hline
7 & Kiểm tra biên lai & Xác minh tích hợp & Biên lai mới được tạo trong tab "Biên lai thanh toán" & \\
\hline
}
{Test case pass khi thanh toán được ghi nhận chính xác và tạo biên lai tương ứng}
{Lead: Nguyễn Văn A, Members: Trần Thị B}
{03/01/2026}

\testcase
{TC-012}
{Kế toán tạo QR code thanh toán}
{Kiểm tra chức năng tạo mã QR cho thanh toán chuyển khoản}
{Functional testing - Kiểm thử tính năng QR payment}
{
1 & Trong tab "Thanh toán ngoại tuyến" & Chọn căn hộ & Hiển thị hóa đơn chưa thanh toán & \\
\hline
2 & Chọn hóa đơn cần thanh toán & Đánh dấu & Checkbox được chọn & \\
\hline
3 & Nhấn "Tạo QR Code" & Mở dialog QR & Dialog hiển thị thông tin thanh toán & \\
\hline
4 & Kiểm tra thông tin QR & Xác minh dữ liệu & Hiển thị: Số tài khoản, số tiền, nội dung chuyển khoản & \\
\hline
5 & Kiểm tra mã QR & Xác minh QR code & QR code được tạo và hiển thị chính xác & \\
\hline
6 & Nhấn "In QR Code" & Kiểm tra print & Hệ thống mở dialog in với QR code & \\
\hline
}
{Test case pass khi QR code được tạo chính xác với đầy đủ thông tin thanh toán}
{Lead: Trần Thị B, Members: Lê Văn C}
{03/01/2026}

\newpage
\section{Test Cases - Chức năng Quản trị (Admin/Manager)}

\testcase
{TC-013}
{Admin quản lý tài khoản người dùng}
{Kiểm tra chức năng CRUD cho tài khoản người dùng}
{Functional testing - Kiểm thử quản lý accounts}
{
1 & Đăng nhập với tài khoản Admin & Truy cập hệ thống & Dashboard admin được hiển thị & \\
\hline
2 & Nhấn vào tab "Tài khoản" & Mở trang quản lý & Hiển thị danh sách tài khoản với bộ lọc theo role & \\
\hline
3 & Nhấn "Thêm tài khoản" & Mở dialog tạo mới & Form hiển thị: Username, Password, Role & \\
\hline
4 & Điền thông tin và lưu & Tạo tài khoản mới & Toast thành công, tài khoản xuất hiện trong danh sách & \\
\hline
5 & Chỉnh sửa tài khoản & Kiểm tra update & Dialog edit hiển thị, cho phép sửa thông tin & \\
\hline
6 & Xóa tài khoản & Kiểm tra delete & Dialog xác nhận hiển thị, sau khi xác nhận tài khoản bị xóa & \\
\hline
}
{Test case pass khi tất cả CRUD operations cho accounts hoạt động chính xác}
{Lead: Hoàng Văn E, Members: Nguyễn Văn A, Phạm Thị D}
{03/01/2026}

\testcase
{TC-014}
{Admin quản lý cư dân}
{Kiểm tra chức năng CRUD cho thông tin cư dân}
{Functional testing - Kiểm thử quản lý residents}
{
1 & Nhấn vào tab "Cư dân" & Mở trang quản lý & Hiển thị danh sách cư dân với thông tin cơ bản & \\
\hline
2 & Tìm kiếm cư dân theo tên & Kiểm tra search & Danh sách được lọc theo từ khóa & \\
\hline
3 & Nhấn "Thêm cư dân" & Mở dialog tạo mới & Form hiển thị: Tên, CCCD, SĐT, Email, Căn hộ & \\
\hline
4 & Điền thông tin hợp lệ & Kiểm tra validation & Form validation đúng format (phone, email, CCCD) & \\
\hline
5 & Lưu cư dân mới & Tạo bản ghi & Toast thành công, cư dân xuất hiện trong danh sách & \\
\hline
6 & Xem chi tiết cư dân & Kiểm tra view & Dialog hiển thị đầy đủ thông tin cư dân & \\
\hline
7 & Cập nhật và xóa & Kiểm tra CRUD & Update và delete hoạt động với confirmation dialog & \\
\hline
}
{Test case pass khi có thể quản lý thông tin cư dân với validation chính xác}
{Lead: Phạm Thị D, Members: Lê Văn C}
{03/01/2026}

\testcase
{TC-015}
{Admin quản lý căn hộ}
{Kiểm tra chức năng CRUD cho thông tin căn hộ}
{Functional testing - Kiểm thử quản lý apartments}
{
1 & Nhấn vào tab "Căn hộ" & Mở trang quản lý & Hiển thị danh sách căn hộ với lọc theo tòa nhà & \\
\hline
2 & Lọc theo tòa nhà & Kiểm tra filter & Danh sách chỉ hiển thị căn hộ của tòa đã chọn & \\
\hline
3 & Nhấn "Thêm căn hộ" & Mở dialog tạo & Form hiển thị: Mã căn hộ, Tầng, Diện tích, Tòa nhà & \\
\hline
4 & Nhập mã căn hộ trùng & Kiểm tra validation & Hiển thị lỗi "Mã căn hộ đã tồn tại" & \\
\hline
5 & Nhập mã căn hộ mới & Tạo căn hộ & Toast thành công, căn hộ được thêm vào danh sách & \\
\hline
6 & Chỉnh sửa thông tin căn hộ & Kiểm tra update & Cập nhật thành công với toast notification & \\
\hline
7 & Xóa căn hộ chưa có cư dân & Kiểm tra delete & Xóa thành công với confirmation & \\
\hline
}
{Test case pass khi quản lý căn hộ hoạt động với validation và constraints chính xác}
{Lead: Nguyễn Văn A, Members: Hoàng Văn E}
{03/01/2026}

\testcase
{TC-016}
{Manager quản lý kế toán}
{Kiểm tra chức năng CRUD cho tài khoản kế toán}
{Functional testing - Kiểm thử quản lý accountants}
{
1 & Đăng nhập với tài khoản Manager & Truy cập hệ thống & Dashboard manager được hiển thị & \\
\hline
2 & Nhấn vào tab "Kế toán" & Mở trang quản lý & Hiển thị danh sách kế toán & \\
\hline
3 & Nhấn "Thêm kế toán" & Mở dialog tạo & Form hiển thị thông tin cơ bản của kế toán & \\
\hline
4 & Điền thông tin và lưu & Tạo bản ghi mới & Kế toán mới được tạo với tài khoản tương ứng & \\
\hline
5 & Kiểm tra trong tab "Tài khoản" & Xác minh tích hợp & Tài khoản với role Accountant được tạo tự động & \\
\hline
6 & Cập nhật thông tin kế toán & Kiểm tra update & Thông tin được cập nhật chính xác & \\
\hline
}
{Test case pass khi quản lý kế toán hoạt động với liên kết accounts đúng}
{Lead: Lê Văn C, Members: Trần Thị B, Phạm Thị D}
{03/01/2026}

\newpage
\section{Test Cases - Tích hợp và Bảo mật}

\testcase
{TC-017}
{Kiểm tra token expiration và refresh}
{Xác minh xử lý khi token hết hạn}
{Security testing - Kiểm thử session management}
{
1 & Đăng nhập vào hệ thống & Tạo session & Token được lưu trong localStorage & \\
\hline
2 & Chờ token hết hạn (mock) & Kiểm tra expiration & Token hết hiệu lực & \\
\hline
3 & Thực hiện một API call & Kiểm tra xử lý & Hệ thống phát hiện token invalid & \\
\hline
4 & Kiểm tra redirect & Xác minh hành vi & User được redirect về trang login & \\
\hline
5 & Kiểm tra thông báo & Xác minh UX & Toast hiển thị "Phiên đăng nhập hết hạn" & \\
\hline
}
{Test case pass khi hệ thống xử lý token expiration một cách graceful}
{Lead: Hoàng Văn E, Members: Nguyễn Văn A}
{03/01/2026}

\testcase
{TC-018}
{Kiểm tra CORS và API security}
{Xác minh cấu hình bảo mật API}
{Security testing - Kiểm thử API security}
{
1 & Gọi API từ domain không được phép & Kiểm tra CORS & Request bị từ chối với CORS error & \\
\hline
2 & Gọi API không có token & Kiểm tra authentication & Response 401 Unauthorized & \\
\hline
3 & Gọi API với token giả mạo & Kiểm tra validation & Response 401 Unauthorized & \\
\hline
4 & Gọi API với token hợp lệ nhưng sai role & Kiểm tra authorization & Response 403 Forbidden & \\
\hline
5 & Gọi API với token và role đúng & Kiểm tra success case & Response 200 OK với dữ liệu chính xác & \\
\hline
}
{Test case pass khi tất cả các cơ chế bảo mật API hoạt động đúng}
{Lead: Nguyễn Văn A, Members: Hoàng Văn E, Lê Văn C}
{03/01/2026}

\testcase
{TC-019}
{Kiểm tra tính toán hóa đơn end-to-end}
{Kiểm tra toàn bộ flow từ nhập chỉ số đến tạo hóa đơn}
{Integration testing - Kiểm thử tích hợp}
{
1 & Kế toán nhập chỉ số công tơ & Tạo dữ liệu đầu vào & Chỉ số điện/nước được lưu cho căn hộ A & \\
\hline
2 & Kế toán tạo/cập nhật phí dịch vụ & Thiết lập đơn giá & Đơn giá điện, nước, và phí khác được cấu hình & \\
\hline
3 & Kế toán chạy tính hóa đơn & Thực hiện tính toán & Hệ thống tính: (chỉ số mới - chỉ số cũ) × đơn giá & \\
\hline
4 & Kiểm tra hóa đơn được tạo & Xác minh kết quả & Hóa đơn hiển thị với tổng tiền chính xác & \\
\hline
5 & Cư dân đăng nhập và xem hóa đơn & Kiểm tra hiển thị & Cư dân A thấy hóa đơn mới trong tab "Hóa đơn" & \\
\hline
6 & Kế toán xử lý thanh toán & Hoàn tất flow & Hóa đơn chuyển sang "Đã thanh toán", biên lai được tạo & \\
\hline
7 & Cư dân xem lịch sử thanh toán & Xác minh end-to-end & Giao dịch xuất hiện trong tab "Thanh toán" & \\
\hline
}
{Test case pass khi toàn bộ flow hoạt động liền mạch từ đầu đến cuối}
{Lead: Lê Văn C, Members: Tất cả members}
{03/01/2026}

\newpage
\section{Test Cases - Giao diện và Trải nghiệm Người dùng}

\testcase
{TC-020}
{Kiểm tra responsive design}
{Xác minh giao diện hoạt động tốt trên nhiều kích thước màn hình}
{UI/UX testing - Kiểm thử responsive}
{
1 & Mở ứng dụng trên desktop (1920×1080) & Kiểm tra layout & Giao diện hiển thị đầy đủ, sidebar và content cân đối & \\
\hline
2 & Thu nhỏ browser xuống tablet (768px) & Kiểm tra breakpoint & Layout điều chỉnh, grid columns thay đổi phù hợp & \\
\hline
3 & Mở trên mobile (375px) & Kiểm tra mobile view & Cards và tables responsive, không bị overflow & \\
\hline
4 & Kiểm tra các dialog/modal & Xác minh responsive & Dialog hiển thị tốt trên mọi kích thước & \\
\hline
5 & Kiểm tra navigation & Xác minh usability & Sidebar có thể collapse trên mobile & \\
\hline
}
{Test case pass khi giao diện hoạt động tốt trên desktop, tablet, và mobile}
{Lead: Trần Thị B, Members: Phạm Thị D}
{03/01/2026}

\testcase
{TC-021}
{Kiểm tra color scheme và accessibility}
{Xác minh màu sắc và khả năng tiếp cận}
{UI/UX testing - Kiểm thử accessibility}
{
1 & Kiểm tra color contrast & Đảm bảo readability & Tất cả text có contrast ratio >= 4.5:1 & \\
\hline
2 & Kiểm tra blue gradient & Xác minh brand & Gradient xanh xuất hiện nhất quán trên headers, cards & \\
\hline
3 & Kiểm tra trạng thái interactive & Xác minh feedback & Buttons có hover, active, disabled states rõ ràng & \\
\hline
4 & Kiểm tra color coding & Xác minh logic & Đỏ=chưa thanh toán, xanh=đã thanh toán, cam=pending & \\
\hline
5 & Navigation bằng keyboard & Kiểm tra accessibility & Tab order hợp lý, focus visible rõ ràng & \\
\hline
}
{Test case pass khi color scheme nhất quán và đáp ứng tiêu chuẩn accessibility}
{Lead: Phạm Thị D, Members: Trần Thị B}
{03/01/2026}

\testcase
{TC-022}
{Kiểm tra loading states và error handling}
{Xác minh hiển thị loading và xử lý lỗi}
{UI/UX testing - Kiểm thử user feedback}
{
1 & Thực hiện API call & Kiểm tra loading & Spinner hoặc skeleton loader hiển thị & \\
\hline
2 & Mô phỏng slow network & Kiểm tra timeout & Loading indicator vẫn hiển thị, không bị treo & \\
\hline
3 & Mô phỏng API error & Kiểm tra error state & Toast error hiển thị với message rõ ràng & \\
\hline
4 & Mô phỏng network offline & Kiểm tra offline mode & Thông báo "Không có kết nối mạng" & \\
\hline
5 & Kiểm tra empty states & Xác minh UX & Empty state với icon và message thân thiện & \\
\hline
}
{Test case pass khi tất cả loading và error states được xử lý tốt}
{Lead: Nguyễn Văn A, Members: Lê Văn C}
{03/01/2026}

\testcase
{TC-023}
{Kiểm tra toast notifications}
{Xác minh hệ thống thông báo ngắn hạn}
{UI/UX testing - Kiểm thử notifications}
{
1 & Thực hiện thao tác thành công & Kiểm tra success toast & Toast màu xanh với icon check hiển thị & \\
\hline
2 & Thực hiện thao tác thất bại & Kiểm tra error toast & Toast màu đỏ với icon X hiển thị & \\
\hline
3 & Kiểm tra vị trí hiển thị & Xác minh placement & Toast hiển thị ở góc phù hợp, không che nội dung & \\
\hline
4 & Kiểm tra auto-dismiss & Xác minh timing & Toast tự động ẩn sau 3-5 giây & \\
\hline
5 & Kiểm tra multiple toasts & Xác minh stacking & Nhiều toast xếp chồng không chồng lấn & \\
\hline
}
{Test case pass khi toast notifications hiển thị rõ ràng và không gây phiền toái}
{Lead: Trần Thị B, Members: Phạm Thị D}
{03/01/2026}

\newpage
\section{Test Cases - Performance và Scalability}

\testcase
{TC-024}
{Kiểm tra performance với dữ liệu lớn}
{Xác minh hiệu năng khi xử lý nhiều bản ghi}
{Performance testing - Kiểm thử hiệu năng}
{
1 & Load danh sách 1000 căn hộ & Kiểm tra rendering & Danh sách hiển thị mượt mà trong < 2 giây & \\
\hline
2 & Tìm kiếm trong danh sách lớn & Kiểm tra search performance & Kết quả hiển thị ngay lập tức (< 500ms) & \\
\hline
3 & Lọc và sắp xếp dữ liệu lớn & Kiểm tra filtering & Filter/sort hoạt động nhanh, không lag & \\
\hline
4 & Pagination với nhiều trang & Kiểm tra pagination & Chuyển trang mượt mà, không bị chậm & \\
\hline
5 & Load nhiều tab cùng lúc & Kiểm tra memory & Không bị memory leak khi switch tabs & \\
\hline
}
{Test case pass khi ứng dụng hoạt động mượt mà với dữ liệu lớn}
{Lead: Hoàng Văn E, Members: Nguyễn Văn A}
{03/01/2026}

\testcase
{TC-025}
{Kiểm tra concurrent users}
{Xác minh hệ thống xử lý nhiều người dùng cùng lúc}
{Performance testing - Kiểm thử đồng thời}
{
1 & 10 users đăng nhập cùng lúc & Kiểm tra auth load & Tất cả users đăng nhập thành công & \\
\hline
2 & Multiple users cập nhật dữ liệu & Kiểm tra race condition & Không có data conflict, version control hoạt động & \\
\hline
3 & Concurrent payment processing & Kiểm tra transaction & Mỗi payment được xử lý đúng, không bị duplicate & \\
\hline
4 & Nhiều users import CSV & Kiểm tra queue & Các import được xử lý tuần tự hoặc parallel an toàn & \\
\hline
}
{Test case pass khi hệ thống xử lý concurrent operations chính xác}
{Lead: Lê Văn C, Members: Hoàng Văn E}
{03/01/2026}

\newpage
\section{Kết luận}

\subsection{Tổng kết}
Tài liệu này đã mô tả 25 test cases quan trọng cho hệ thống BlueMoon, bao gồm:
\begin{itemize}
    \item 3 test cases về xác thực và phân quyền
    \item 3 test cases về chức năng cư dân
    \item 6 test cases về chức năng kế toán
    \item 4 test cases về chức năng quản trị
    \item 3 test cases về tích hợp và bảo mật
    \item 4 test cases về giao diện và trải nghiệm người dùng
    \item 2 test cases về hiệu năng
\end{itemize}

\subsection{Tiêu chí Pass/Fail}
\begin{itemize}
    \item \textbf{Pass}: Tất cả các bước thực hiện đúng kết quả mong đợi
    \item \textbf{Failed}: Bất kỳ bước nào không đạt kết quả mong đợi
    \item \textbf{Pending}: Test case chưa được thực hiện
\end{itemize}

\subsection{Môi trường kiểm thử}
\begin{itemize}
    \item \textbf{Frontend}: React 18 + TypeScript + Tailwind CSS
    \item \textbf{Backend}: Python FastAPI
    \item \textbf{Database}: PostgreSQL
    \item \textbf{Browsers}: Chrome (latest), Firefox (latest), Safari (latest)
    \item \textbf{Devices}: Desktop (1920×1080), Tablet (768×1024), Mobile (375×667)
\end{itemize}

\subsection{Lịch trình kiểm thử}
\begin{itemize}
    \item \textbf{Phase 1} (Week 1): Test cases TC-001 đến TC-008
    \item \textbf{Phase 2} (Week 2): Test cases TC-009 đến TC-016
    \item \textbf{Phase 3} (Week 3): Test cases TC-017 đến TC-025
    \item \textbf{Regression testing}: Thực hiện sau mỗi sprint
\end{itemize}

\subsection{Định nghĩa vai trò trong nhóm kiểm thử}
\begin{itemize}
    \item \textbf{Nguyễn Văn A}: Test Lead - Chịu trách nhiệm authentication, authorization, resident features
    \item \textbf{Trần Thị B}: Tester - Chịu trách nhiệm UI/UX testing, notifications
    \item \textbf{Lê Văn C}: Tester - Chịu trách nhiệm accountant features, integration testing
    \item \textbf{Phạm Thị D}: Tester - Chịu trách nhiệm admin features, accessibility
    \item \textbf{Hoàng Văn E}: Tester - Chịu trách nhiệm performance testing, security
\end{itemize}

\subsection{Tài liệu tham khảo}
\begin{itemize}
    \item GitHub Repository: \texttt{https://github.com/kevin715855/BlueMoon}
    \item API Documentation: Backend FastAPI docs
    \item Frontend Components: React TypeScript implementation
    \item Test Data: Sample data trong database test
\end{itemize}

\end{document}
